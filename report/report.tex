\documentclass[a4paper,twoside]{article}
\usepackage[a4paper]{geometry}

% \usepackage[a4paper,showframe]{geometry} % add showframe for debug
% \usepackage{titleps}

\usepackage[pagestyles]{titlesec}
\usepackage[all]{nowidow} % avoid orphan lines

% Extra colours; need to be first
\usepackage[usenames,dvipsnames,svgnames,table]{xcolor}

% Basic packages
\usepackage[utf8]{inputenc}
\usepackage[T1]{fontenc}

% Landscape mode
\usepackage{pdflscape}

% Additional font
\usepackage{FiraMono}

% Bibtex+cite
\usepackage{cite}
\usepackage{url}
\usepackage[nottoc,numbib]{tocbibind}

% URL
\usepackage[
  colorlinks=true,
  urlcolor=blue,
  citecolor=black,
  linkcolor=.
]{hyperref}

% Lists
\usepackage{scrextend}
\addtokomafont{labelinglabel}{\sffamily}
\usepackage[inline]{enumitem} % for enumerate* environment.

% Maths & symbols
\usepackage{mathtools}
\usepackage{amssymb}
\usepackage{pifont}

% Mono font
\usepackage[scaled]{beramono}

% Spacing
\usepackage{xspace}

% Graphics
\usepackage{graphicx}

% Table
\usepackage{booktabs}
\usepackage{tabularx}
\usepackage{multirow}

% Floating placement
\usepackage{placeins}
% use \FloatBarrier before a \section to ensure all floating are displayed
% before the new section

% Caption styling
\usepackage[justification=centering]{caption}

% Additional colours
\definecolor{c1}{HTML}{006C71}
\definecolor{c2}{HTML}{005155}
\definecolor{c3}{HTML}{FF8928}
\definecolor{c4}{HTML}{E86900}
\colorlet{ImportantCode}{ForestGreen}
\colorlet{ImportantCode2}{RubineRed}
\colorlet{ImportantCode3}{RedOrange}

% TIKZ & related packages/settings
\usepackage{tikz}
\usepackage{pgfplots}
\usetikzlibrary{shapes,fit,backgrounds,arrows,positioning,chains,patterns}
% \pgfplotsset{compat = 1.14}

% for caching tikz pictures
% \usetikzlibrary{external}
% \tikzexternalize[prefix=tikz-figures/]
% \pgfrealjobname{tikz}


% Listings
\usepackage{listings,multicol}
\usepackage{lstscala}


\lstdefinelanguage{PureScala}{ % Using `Scala` result in a infinite recursion
  style=scala-color,
  morekeywords={[2]Unit,Boolean,Byte,Int,BigInt,String,Char,true,false},
  %keywordstyle={[2]\color{blue!30!darkgray}\bfseries}
}

\newcommand{\Inline}[1]{\lstinline[basicstyle=\ttfamily,columns=fixed]|#1|}
\newcommand{\InlineS}[1]{\lstinline[language=PureScala,columns=fixed]|#1|}
% %\newcommand{\inlineScala}[1]{\lstinline[language=MyScala,breaklines=true,breakatwhitespace=true]|#1|}

% Use code in description item, Based on http://tex.stackexchange.com/a/181325/77356
\newcommand*{\lstitem}[1]{
  \setbox0\hbox{\textbf{\Inline{#1}}}
  \item[\usebox0]
}

% %\lstset{aboveskip=5pt,belowskip=10pt}
% \lstset{captionpos=b,abovecaptionskip=1em}

% For long listings
\lstdefinestyle{LongCode}{
  %aboveskip=1ex,
  % basicstyle=\small\ttfamily,
  basicstyle=\scriptsize\ttfamily,
  % basicstyle=\tiny\ttfamily,
  %belowskip=1ex,
  breaklines=true,
  postbreak=\raisebox{0ex}[0ex][0ex]{\ensuremath{\color{red}\hookrightarrow}},
  breakautoindent=false,
  %breakatwhitespace=false,
  %columns=fullflexible,
  multicols=2,
  framerule=0pt,
  framexrightmargin=0em,
  framexleftmargin=0em,
  % numbers=left,
  % numberstyle=\footnotesize\sffamily,
  tabsize=2
}


% Env for short code: no label, no caption and "inlined" with a box
\lstnewenvironment{ShortCode}[1]
    {
      \lstset{
        language=#1,
        frame=trBL % box the frame & double line on bottom and left side
      }
    }
    {
    }

% Env for longer code: label, caption and "floating" with a box
\lstnewenvironment{Code}[3]
    {
      \lstset{
        language=#1,
        label={#2},
        caption=#3,
        captionpos=b, % bottom
        float=!h,     % "here"
        frame=trBL,   % box the frame & double line on bottom and left side
        % frameround=tttt
      }
    }
    {
      % DEBUG:
      % 1: #1 \\
      % 2: #2 \\
      % 3: #3
    }



% Tables
% \newcommand{\heading}[1]{\multicolumn{1}{c}{\textbf{#1}}}
% \newcommand{\vheading}[1]{\rotatebox[origin=c]{90}{~\textbf{#1}~}}


% To centre table too wide
% credit: http://tex.stackexchange.com/a/27099/77356
\makeatletter
\newcommand*{\centerfloat}{%
  \parindent \z@
  \leftskip \z@ \@plus 1fil \@minus \textwidth
  \rightskip\leftskip
  \parfillskip \z@skip}
\makeatother


% General styling
%\let\oldsection\section
%\renewcommand\section{\cleardoublepage\oldsection}

% For some reason page margin are swapped when having a title page with a
% different geometry. We fix that manually by 1) swapping the margin and
% 2) swapping the page number position (page style)
% credit: http://tex.stackexchange.com/a/36016/77356
\makeatletter
\newcommand*{\flipmargins}{%
  \clearpage
  \setlength{\@tempdima}{\oddsidemargin}%
  \setlength{\oddsidemargin}{\evensidemargin}%
  \setlength{\evensidemargin}{\@tempdima}%
  \if@reversemargin
    \normalmarginpar
  \else
    \reversemarginpar
  \fi
}
\makeatother

\newpagestyle{mystyle_no_header}{
  \setfoot[\thepage][][]{}{}{\thepage}
}

\newpagestyle{mystyle}{
  % \sethead[][][\firsttitlemarks\ifthesubsection{\thesubsection~\subsectiontitle}{\thesection~\sectiontitle}]{\firsttitlemarks\ifthesubsection{\thesubsection~\subsectiontitle}{\thesection~\sectiontitle}}{}{}
  \sethead[\thesection~\sectiontitle][][]{}{}{\thesection~\sectiontitle}
  % \setfoot[][][\thepage]{\thepage}{}{}
  \setfoot[\thepage][][]{}{}{\thepage}
}

% Additional macros
\newcommand{\TODO}[1]{\textcolor{YellowOrange}{(#1)}} % for inline TODO
% \newcommand{\TODO}[1]{\underline{(#1)}} % for inline TODO, PRINTING
\newcommand{\GenC}{\emph{GenC}\xspace}
\newcommand{\URL}[2]{#2:\xspace\href{#1}{#1}}
\newcommand{\RefSec}[1]{Section~\ref{#1}}
\newcommand{\RefTable}[1]{Table~\ref{#1}}
\newcommand{\RefApp}[1]{Appendix~\ref{#1}}
\newcommand{\RefFig}[1]{Figure~\ref{#1}}
\newcommand{\RefCode}[1]{Listing~\ref{#1}}
\newcommand{\BigO}[1]{\mathcal{O}(#1)}




%%%

\title{Verification of complex systems in Stainless}

\date{
  {\small Version 0.1}\\
  December 2017
}

\author{Romain Ruetschi}

\begin{document}

\newgeometry{centering}
\pagenumbering{gobble}
\maketitle

\vfill

\begin{abstract}

TODO

\end{abstract}

\vfill

\begin{center}
    Master Thesis Project under the supervision of \\
    Prof. Viktor Kuncak \& Dr. Jad Hamza \\
    Lab for Automated Reasoning and Analysis LARA - EPFL
\end{center}

\begin{center}
    \includegraphics[width = 40mm]{res/epfl-logo}
\end{center}

\clearpage\null\newpage


\restoregeometry              %% restore the layout
\flipmargins
\pagestyle{mystyle_no_header}
\pagenumbering{arabic}

\tableofcontents
% \lstlistoflistings


\clearpage
\pagestyle{mystyle}

\section{Introduction}

TODO

\subsection*{Related Works}

TODO

\clearpage
\section{Motivation}
\label{motivation}


TODO

\clearpage
\section{Verifiying Actor Systems}
\label{actors}

\subsection{The Actor Model}

TODO

\subsubsection{Message}

In our framework, messages are modelled as concrete subclasses of the abstract class 
\InlineS{Msg}.

\subsubsection{Actor Reference}

An actor can be referenced by its \InlineS{ActorRef}.

\subsubsection{Message in Flight}

In-flight messages are represented as a product type of the destination \InlineS{ActorRef} 
and the \InlineS{Msg}.

\subsubsection{Behavior}

A behavior specifies both the current state of an actor, and how this one should 
process the next incoming message. In our framework, these are modelled as a subclass 
of the abstract class \InlineS{Behavior}, which defines a single abstract method 
\InlineS{processMsg}, to be overriden for each defined behavior.

\begin{Code}{PureScala}{lst:Behavior}{Behavior in PureScala}
abstract class Msg

abstract class ActorRef {
  def !(msg: Msg)(implicit ctx: ActorContext): Unit = {
    ctx.send(this, msg)
  }
}

case class Packet(dest: ActorRef, payload: Msg)

case class ActorContext(
  self: ActorRef,
  var toSend: List[Packet]
) {
  def send(to: ActorRef, msg: Msg): Unit = {
    toSend = toSend :+ Packet(to, msg)
  }
}

abstract class Behavior {
  def processMsg(msg: Msg)(implicit ctx: ActorContext): Behavior
}
\end{Code}

\subsubsection{Actor Context}

As mentioned above, when a message is delivered to an actor, the latter is provided with
a context, which holds a reference to itself, as well as a list of \InlineS{Packet}s 
to send.

\subsubsection{Transition}

\begin{Code}{PureScala}{lst:ActorSystem}{Transition in PureScala}
case class Transition(
  from: ActorRef,
  to: ActorRef,
  msg: Msg,
  newBehavior: Behavior,
  toSend: List[Packet]
)
\end{Code}

\subsubsection{Actor System}


\begin{Code}{PureScala}{lst:ActorSystem}{Actor System in PureScala}
case class ActorSystem(
  behaviors: CMap[ActorRef, Behavior],
  inboxes: CMap[(ActorRef, ActorRef), List[Msg]],
  trace: List[Transition]
) {
  def step(from: ActorRef, to: ActorRef): ActorSystem = /* ... */
}
\end{Code}


\subsection{Proving Invariants}
\label{invariants}

TODO

\subsection{Reasoning About Traces}
\label{traces}


TODO

\clearpage
\section{Biparty Communication Protocols}
\label{biparty}


TODO

\clearpage
\section{Conclusion}
\label{conclusion}

TODO

\clearpage
\section{Future Works}
\label{future_works}

\appendix

TODO

\clearpage

\bibliographystyle{ieeetr}
\bibliography{report}

\end{document}
